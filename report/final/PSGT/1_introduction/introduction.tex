
% this file is called up by thesis.tex
% content in this file will be fed into the main document

%: ----------------------- introduction file header -----------------------
\chapter{Introduction} \label{introduction}

% the code below specifies where the figures are stored
\ifpdf
    \graphicspath{{1_introduction/figures/PNG/}{1_introduction/figures/PDF/}{1_introduction/figures/}}
\else
    \graphicspath{{1_introduction/figures/EPS/}{1_introduction/figures/}}
\fi

% ----------------------------------------------------------------------
%: ----------------------- introduction content ----------------------- 
% ----------------------------------------------------------------------

The overall goal of this report is to investigate what makes a good story, and then investigate methods for algorithmically generating them. The question of "Why?" may be asked. What makes stories important enough to warrant this kind of work?

Stories are widely believed to be incredibly useful for fostering an understanding of the shared human experience and questions of existence (\Citealt{eder2010life}), for educational communication (\Citealt{birch1996says}) and at their most incisive, contributing to social and political change (\Citealt{fuertes2012storytelling}). This is all in addition to the entertainment value we all gain from stories, whether they be written, recorded, or shared via word of mouth. These stories develop whole ideologies and cultures; one need only look at religions for evidence, and we have records of written literature dating back to 2600 BCE (\Citealt{grimbly2013encyclopedia}).

For these reasons, we believe this research to be incredibly worthwhile. If we can aid or expedite the writing process with the introduction of procedurally generated contributions, this has the potential to serve as a tool to foster even more creativity in human writers. Inspiration for this is drawn from Google Deep Mind (\href{https://deepmind.com/}{https://deepmind.com/}) and their Alpha Go (\Citealt{silver2017mastering}) research: an artificial intelligence (AI) program designed to tackle the ancient Chinese board game of Go (\href{https://www.cs.cmu.edu/~wjh/go/rules/Chinese.html}{https://www.cs.cmu.edu/~wjh/go/rules/Chinese.html}). While much of the detail of the research they conducted is beyond the scope of ours, one takeaway from their research was that humans learned and became better for having played the AI program, in the same way they might improve while playing a superior human. It is hoped that the application developed as a result of this research could serve in a similar manner, seeing improvements in human writers with the help of an AI writer, as well as the AI's productions themselves.

\vspace{10mm}

% here we declare a new section

\section{Aims and Objectives} 

Two main questions to be addressed are: what is a good story, and how may they be procedurally generated? In order to address these questions, this work focuses on the following areas:

\begin{itemize}

\item An examination of stories from a linguistic perspective. Older research or that which does not concern itself so much with technology. This includes things like story structures, character development, plot devices and so on. Creating a perfect definition for a "good story" may be implausible, but each strand and structural element that we can understand and incorporate gets us closer to understanding the nature of stories.

\item Algorithms for formulating these elements in such a way as to generate stories. We will look at a number of candidates, evaluate each of them their general merits and then decide which one shows the greatest potential. 

\item Technologies that utilise these algorithms, researching advancements that have been made from early stage implementations up through state-of-the-art. This will be done with the product in mind, choosing which way to approach its creation.

\item From here we will produce a prototype product which will generate stories and allow human users to interact and modify these stories as they go. In the same way that a human-computer combination has proved more effective than a human alone in chess (\Citealt{michie1972programmer}), we aim to create an interactive co-writing experience.

\end{itemize}



% here we declare a new section
\section{Methodology} 

In order to address these questions we will follow the methodology outlined in sequence above, before designing the system itself and implementing the product. Afterwards, we will reflect and perform various types of evaluation on the productions of the system, noting that it is not primarily intended to be completely autonomous but rather with human interaction. Evaluation of the output will include known language generation metrics as well as human review.
 
 
% here we declare a new section
\section{Research Contribution}

A comprehensive history and discussion of what makes stories a worthwhile endeavour; how their quality and elements may be defined; and ultimately  a prototype that demonstrates the potential of automated techniques to act as an aid for the human creative in the story writing process.

\section{Report Outline} 

The remaining chapters of this report are as follows:


\emph{Chapter Two} outlines and discusses the history and related research to this topic, from linguistic and technological perspectives; \emph{Chapter Three} relates to the design of my chosen system and the choices that were made with regard to models, architecture etc; \emph{Chapter Four} presents the implementation of the system, portions of interesting code, struggles that were faced and how I overcame them; \emph{Chapter Five} deals with the evaluation of productions from the system, both objective metrics and subjective human review; \emph{Chapter Six} draws conclusions, evaluates my satisfaction and areas for improvement, and suggests possible future works and research.

%\vspace{5 mm}

%A series of documents have been included in the Appendix section of this dissertation. These are:
%
%\begin{itemize}
%\item \emph{Appendix A} outlines . . .
%
%\item \emph{Appendix B} presents . . .
%
%\item \emph{Appendix C} includes . . .
%\end{itemize} 
%
%\vspace{5 mm}
%
%Attached to this dissertation is a CD containing the following items:
%
%\begin{itemize}
%\item \emph{folder 1}: . . .
%
%\item \emph{folder 2}: . . .
%
%\end{itemize}
