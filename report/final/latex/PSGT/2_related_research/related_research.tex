% this file is called up by thesis.tex
% content in this file will be fed into the main document

%: ----------------------- name of chapter  -------------------------

\chapter{Related Research} % top level followed by section, subsection

%: ----------------------- paths to graphics ------------------------

% change according to folder and file names
\ifpdf
    \graphicspath{{2_related_research/figures/PNG/}{2_related_research/figures/PDF/}{2_related_research/figures/}}
\else
    \graphicspath{{2_related_research/figures/EPS/}{2_related_research/figures/}}
\fi

%: ----------------------- contents from here ------------------------

Much research has been undertaken on this topic, with a variety of approaches.


\section{Stories}

It is important for us to understand the most basic, linguistic concepts before we move on to algorithmic productions. What are the elements that make up a good story? What constitutes a good story? What even is a story? 

According to Webster (\Citealt{dictionary2002merriam}), a \emph{Story} is: 
\begin{quote}
An account of incidents or events, [either] regarding the facts pertinent to a situation in question, [or] a fictional narrative.
\end{quote}

With a \emph{narrative} being:
\begin{quote}
A way of presenting or understanding a situation or series of events that reflects and promotes a particular point of view or set of values.
\end{quote}

This is naturally a broad set of definitions, but it does give us some cues. From a story, we would expect a series of events related to each other in some way, either to describe a situation or promote a certain worldview or message. That is, there is an overall connection and cohesiveness to the piece. Literary critics have posited several interpretations or generalisations: that stories begin in equilibrium before being disrupted, and ultimately involve a journey back to equilibrium (\Citealt{todorov1969structural}), but yet more argue that there can be no "correct" definition determined (\Citealt{sullivan2002reception}). The fields of Literary Theory and Narratology emerged in an attempt to dissect and formulate stories, which we will discuss more later.

I first set out to examine commonly used techniques used in various aspects of storytelling. These will perhaps be more relevant in evaluating the productions of my system, rather than guiding them too much, depending on the level of autonomy, and the rigidity of training and generation it has.

\subsection{Structure}

Structure describes the underlying framework of a story and, as the highest level of a story planning process, was first to be investigated. 

The classic structure would be the three acts. This dates back to Aristotle in 400BC, describing a story as having three parts: a beginning, an end and a middle (\Citealt{mack1980norton}), and is still popular today. It is commonly depicted something like (Figure \ref{3-act-structure}).
 
\figuremacroW{3-act-structure}{Three Act Structure}{Breakdown of three act structure. [Source: (\Citealt{3-act-structure-image})]}{0.8}

Stories are broken town into distinct sections: setup and exposition, rising action and confrontation, climax and resolution (\Citealt{trottier1998screenwriter}). This is a bit more versatile, which is something to consider for the later steps of “filling in the gaps” with our algorithmic approach. A balance must be struck between: providing some structure so that the generation has some level of cogency, but also allowing flexibility so that not all stories are the same.

Other popular methods include the Hero’s Journey (\Citealt{campbell2008hero}), which describes a cycle of sorts: a call to adventure, crossing from the known to the unknown, transformation etc. This is a more granular structure, most known for its application in Star Wars. See (Figure \ref{heros-journey}).
 
\figuremacroW{heros-journey}{The Hero's Journey}{A cycle of the hero's journey. [Source: (\Citealt{vogler1985practical})]}{0.8}

However, that is not to say these are the only structures that must be followed, nor that there must be that traditional arc. Some authors have examined spiral, fractal and explosive patterns in literature (\Citealt{alison2019meander}), rejecting the historical, structural norms. It is tempting to declare adherence to a structure irrelevant, but patterns do remain, so this cannot be ignored entirely.

\subsection{Plot}
\label{story_plot}

Delving deeper into the story, plot makes up those events which are significant, have consequences and make a difference to the story (\Citealt{dibell1999elements}). If we examine our three act structure figure from earlier (Figure \ref{3-act-structure}), we see ticks along the line of the story, larger incidents which have an impact.

Indeed, a scene or series of events may be memorable and iconic, but if they do not serve as major events that progress the overall narrative, then they do not constitute plot (\Citealt{alcorn2014know}). Following on from the three-act structure, plot points would be used to connect the acts to each other, for example: our protagonist is thrust into an unexpected situation, they face a setback and it seems all hope is lost, finally they overcome.

For the context of our system, plot points should serve as transitional pieces, advancing the story in some way so we don't simply remain at or revert back to the previous content. This must be handled with care, as too few plot points would be boring, but too many would be bewildering.

\subsection{Setting}

This refers to the time, location and milieu in which the story occurs (\Citealt{lodge2012art}) often referred to as the "world" or "universe", in modern works. 

This serves as the backdrop of our story, and to feel authentic it must be rich with context and history. At their best, settings are so specific that they provide natural associations to the reader (\Citealt{kuntz1993narrative}), setting the mood and plot anticipations.

When generating content, there should be at minimum a consistency of setting, and ideally it should have enough detail to establish a mood that carries through the story.


\section{Narratology}

Having touched on literary theories, I was set on to the work of Vladimir Propp on Narratology (the study of narrative) and his early work in formulating elements of stories. Specifically, his seminal work with Morphology of the Folktale (\Citealt{propp1968morphology}), originally written in 1928. He, along with other Russian formalists, took a modern approach to narratology after Aristotle's ancient theorising.

They distinguished the syuzhet (plot) from the fabula (story). The idea was that the story is the raw material, familiar in many ways already, and it is \emph{defamiliarised} (a term they coined) into the plot, a new organisation and the way the story is told. This goes back to our previous point on Plot (\ref{story_plot}), which pointed out that plot pertains to the information that pushes a story along. They are subtly different concepts.

\subsection{Abstractions}

Propp's work is very relevant to this research, since he has some of the earliest work on abstracting and formulating aspects of stories (specifically Russian folktales).

He first curated a table of possible events at various stages of a story, then associated these with symbols and combined them with functions into what look like mathematical formulas. An example looks something like: (Figure \ref{propp-eg}).
 
\figuremacroW{propp-eg}{Morphology of the Folktale}{Breaking down a story. [Source: (\Citealt{propp1968morphology})]}{0.8}

There were hundreds of these, assembled in a tabular format like so: (Figure \ref{propp-eg-2}).
 
\figuremacroW{propp-eg-2}{Morphology of the Folktale}{Table of stories. [Source: (\Citealt{propp1968morphology})]}{0.8}

This laid groundwork for the development of grammars, while this is perhaps too rigid and systemic to be applicable, as argued by some (\Citealt{dundes1997binary}). 


\section{Formal Language Theory}

(Formal) Grammars were devised as a more generic and granular approach to generating strings of text, by following certain rules, from a certain alphabet (\Citealt{reghizzi2013formal}). Compared to Propp's work on formulating \emph{elements}, this was work being done down to the character level, getting closer to what we would need for algorithmic generation.

Emil Post was one of the early innovators in this area, creating the Post Canonical System in 1943, a string manipulation system for generating instances of a language, from an initial alphabet and rules (\Citealt{post1943formal}). 

\subsection{Grammars}

Noam Chomsky then proposed a set of generative grammars in 1956, classified in the \emph{Chomsky Hierarchy} (\Citealt{chomsky1956three}), with different levels of strictness in their rules. The two efficient and popular types were the Context Free Grammar and Regular Grammar.

Chomsky grammars consist of a finite set of production rules (left-hand side$\,\to\,$right-hand side), where each side consists of a finite sequence of the following symbols:
\begin{itemize}
\item a finite set of nonterminal symbols (indicating a production rule can be applied)
\item a finite set of terminal symbols (indicating no production rule can be applied)
\item a start symbol (a distinguished nonterminal symbol that is not found on any right hand side, and so cannot be produced anyway else)
\end{itemize}

For example:

\begin{equation}
    S \rightarrow AB
\end{equation}\begin{equation}
    S \rightarrow \lambda (empty string)
\end{equation}\begin{equation}
    A \rightarrow aS
\end{equation}\begin{equation}
    B \rightarrow b
\end{equation}

This is a Context Free Grammar (CFG) that could generate a string of letters "a" and "b". 

However, while these are interesting and worth exploring from a historical perspective, formal grammars like this require significant human building and labelling (\Citealt{compton2014tracery}), which is troubling from the perspectives of extensibility and originality.


\section{Natural Language Processing}

Definitions, brief history

\subsection{Neural Networks}

So hot right now.

\subsubsection{Recurrent Neural Networks \& LSTMs}

Better, and best for a while, but still not great.

\subsection{Transformers}

vaswani et al, Attention is all you need, faster, longer range dependencies

\subsubsection{BERT}

Google

\subsubsection{GPT}

First attempt

\subsubsection{GPT-2}

New and improved!



% ---------------------------------------------------------------------------
%: ----------------------- end of thesis sub-document ------------------------
% ---------------------------------------------------------------------------

