% this file is called up by thesis.tex
% content in this file will be fed into the main document

%: ----------------------- name of chapter  -------------------------

\chapter{Related Research} % top level followed by section, subsection

%: ----------------------- paths to graphics ------------------------

% change according to folder and file names
\ifpdf
    \graphicspath{{2_realated_research/figures/PNG/}{2_realated_research/figures/PDF/}{2_realated_research/figures/}}
\else
    \graphicspath{{2_realated_research/figures/EPS/}{2_realated_research/figures/}}
\fi

%: ----------------------- contents from here ------------------------

Much research has been undertaken on this topic, with a variety of approaches.


\section{Stories}

It is important for us to understand the most basic, linguistic concepts before we move on to algorithmic productions. What are the elements that make up a good story? What constitutes a good story? What even is a story?

According to Webster (\Citealt{dictionary2002merriam}), a \emph{Story} is: 
\begin{quote}
An account of incidents or events, [either] regarding the facts pertinent to a situation in question, [or] a fictional narrative.
\end{quote}

With a \emph{narrative} being:
\begin{quote}
A way of presenting or understanding a situation or series of events that reflects and promotes a particular point of view or set of values.
\end{quote}

This is naturally a broad set of definitions, but it does give us some cues. From a story, we would expect a series of events related to each other in some way, either to describe a situation or promote a certain worldview or message. That is, there is an overall connection and cohesiveness to the piece. Literary critics have posited several interpretations or generalisations: that stories begin in equilibrium before being disrupted, and ultimately involve a journey back to equilibrium (\Citealt{todorov1969structural}), but yet more argue that there can be no "correct" definition determined (\Citealt{sullivan2002reception}). The fields of Literary Theory and Narratology emerged in an attempt to dissect and formulate stories, which we will discuss more later.

I first set out to examine commonly used techniques used in various aspects of storytelling.

\subsection{Structure}

Structure describes the underlying framework of a story and, as the highest level of a story planning process, was first to be investigated. 

Popular methods include the Hero’s Journey (\Citealt{campbell2008hero}), which describes a cycle of sorts: a call to adventure, crossing from the known to the unknown, transformation etc. This is a more granular structure, most known for its application in Star Wars.

Another method is the three (or five) act structure (\Citealt{trottier1998screenwriter}), wherein the story is broken down into distinct sections, typically: setup, confrontation, resolution. This is a bit more versatile, which was something to consider for the later steps of “filling in the gaps” with PCG. There was a balance to be struck: providing some structure so that the generation has some level of cogency, but also allowing flexibility so that not all stories are the same.
Following this, a tree-based approach became appealing, at the higher levels resembling more of a novel plan with plot points, and is eventually fleshed out below into sentences and phrases. This is a familiar concept in Computer Science, and it seems like the best way to maintain a coherent story. Some optional user interaction would be ideal – e.g. “Tell me a story about a dog with superpowers” – a kind of “seed”. These could also be generated autonomously.

%(Figure \ref{image02}).
% 
%\figuremacroW{image02}{Title of Figure}{Description. [Source: (\Citealt{mydownloadedimage})]}{0.7}

\subsection{Plot}

You can learn more about Latex by clicking \href{http://en.wikibooks.org/wiki/LaTeX}{here}.

\subsection{Setting}

But how?!


\section{Narratology}

Propp!

\subsection{Abstractions}

Marriage, villain, etc.


\section{Technology}

NLP!

\subsection{NLP}

Definitions, brief history

\subsection{Grammars}

CFG, Context Free, old and unworkable

\subsection{Neural Networks}

So hot right now.

\subsubsection{Recurrent Neural Networks \& LSTMs}

Better, and best for a while, but still not great.

\subsection{Transformers}

vaswani et al, Attention is all you need, faster, longer range dependencies

\subsubsection{BERT}

Google

\subsubsection{GPT}

First attempt

\subsubsection{GPT-2}

New and improved!



% ---------------------------------------------------------------------------
%: ----------------------- end of thesis sub-document ------------------------
% ---------------------------------------------------------------------------

