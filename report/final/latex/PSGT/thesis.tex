
% ----------------------------------------------------------------------------------------------------------------
%                   LATEX TEMPLATE FOR PhD THESIS AT UNIVERSITY OF LIMERICK
% -----------------------------------------------------------------------------------------------------------------

% based on Harish Bhanderi's PhD/MPhil template, then Uni Cambridge
% http://www-h.eng.cam.ac.uk/help/tpl/textprocessing/ThesisStyle/
% corrected and extended in 2007 by Jakob Suckale, then MPI-CBG PhD programme
% and made available through OpenWetWare.org - the free biology wiki
% corrected and extended in2013 by Giuseppe Torre, University of Limerick
% available at:  

%: Style file for Latex
% Most style definitions are in the external file PhDthesisPSnPDF.
% In this template package, it can be found in ./Latex/Classes/
\documentclass[twoside,12pt]{Latex/Classes/PhDthesisPSnPDF}

\usepackage{natbib}
\usepackage{url}
\usepackage{alltt} 
\usepackage{multirow}
\usepackage{color}
\usepackage{tikz}
\usepackage{booktabs}
\usepackage{tablefootnote}
\usepackage{hyperref}
\def\layersep{2.5cm}
\renewcommand\and{\\[\baselineskip]}



\newcommand{\urlwofont}[1]
{
\urlstyle{same}\url{#1}
}

% to display Math
\usepackage{mathtools}

% to Display code
\DeclareFixedFont{\ttb}{T1}{txtt}{bx}{n}{12} % for bold
\DeclareFixedFont{\ttm}{T1}{txtt}{m}{n}{12}  % for normal

\usepackage{color}
\definecolor{deepblue}{rgb}{0,0,0.5}
\definecolor{deepred}{rgb}{0.6,0,0}
\definecolor{deepgreen}{rgb}{0,0.5,0}
\usepackage{listings}
\lstset{language=Python}


%% Define a new 'leo' style for the package that will use a smaller font.
\makeatletter  
\def\url@leostyle{%
  \@ifundefined{selectfont}{\def\UrlFont{\sf}}{\def\UrlFont{\small\ttfamily}}}
\makeatother
%% Now actually use the newly defined style.
\urlstyle{leo}

%: Macro file for Latex
% Macros help you summarise frequently repeated Latex commands.
% Here, they are placed in an external file /Latex/Macros/MacroFile1.tex
% An macro that you may use frequently is the figuremacro (see introduction.tex)
\include{Latex/Macros/MacroFile1}



%: ----------------------------------------------------------------------
%:                  TITLE PAGE: name, degree,..t
% ----------------------------------------------------------------------
% below is to generate the title page with crest and author name

%if output to PDF then put the following in PDF header
\ifpdf  
    \pdfinfo { /Title  (Procedural Story Generation with Transformers)
               /Creator (Niall Dillane)
               /Producer (Niall Dillane)
               /Author (Niall Dillane 13132911@studentmail.ul.ie)
               /CreationDate (D:20200101000000)  %format D:YYYYMMDDhhmmss
               /ModDate (D:YYYYMMDDhhmm)
               /Subject (Procedural Story Generation with Transformers)
               /Keywords (procedural algorithms transformers neural networks) }
    \pdfcatalog { /PageMode (/UseOutlines)
                  /OpenAction (fitbh)  }
\fi


\title{Procedural Story Generation \\ with Transformers}



% ----------------------------------------------------------------------
% The section below defines www links/email for author and institutions
% They will appear on the title page of the PDF and can be clicked
\ifpdf
  \author{\href{mailto: 13132911@studentmail.ul.ie}{Niall Dillane}, \href{mailto: james.patten@ul.ie}{James Patten}}
  \department{\href{http://www.csis.ul.ie}{Computer Science and Information Systems}}
  \university{\href{http://www.ul.ie}{University of Limerick}}
  \crest{\includegraphics[width=7cm]{centre_UL_logo.jpg}}

%\renewcommand{\submittedtext}{change the default text here if needed}
\degree{BSc Computer Systems 2020}
\degreedate{}


% ----------------------------------------------------------------------
       
% turn of those nasty overfull and underfull hboxes
\hbadness=10000
\hfuzz=50pt


%: --------------------------------------------------------------
%:                  FRONT MATTER: dedications, abstract,..
% --------------------------------------------------------------

\begin{document}

%\language{english}



% sets line spacing
\renewcommand\baselinestretch{1.2}
\baselineskip=18pt plus1pt


%: ----------------------- generate cover page ------------------------

\maketitle  % command to print the title page with above variables


%: ----------------------- cover page back side ------------------------
% Your research institution may require reviewer names, etc.
% This cover back side is required by Dresden Med Fac; uncomment if needed.

\newpage

%: ----------------------- abstract ------------------------

% Your institution may have specific regulations if you need an abstract and where it is to be placed in the document. The default here is just after title.
%: Declaration of originality


% Thesis Abstract -----------------------------------------------------


%\begin{abstractslong}    %uncommenting this line, gives a different abstract heading
\begin{abstracts}        %this creates the heading for the abstract page

Procedural content generation (PCG) – the process of generating data algorithmically – is a technique that has applications across a variety of domains. In the research outlined in this report, focus is directed to the use of PCG as a means to generate novel-like stories. The challenge is twofold: research into procedural generation methods, as well as the structure and language of stories, addressing questions such as: “What are the elements of a good story?”. Finally, methods to codify these elements must be investigated, in such a way that they can be used by a procedural generation technique, effectively combining the two disciplines.

\end{abstracts}
%\end{abstractlongs}


% ---------------------------------------------------------------------- 



% Thesis statement of originality -------------------------------------

% Depending on the regulations of your faculty you may need a declaration like the one below. This specific one is from the medical faculty of the university of Dresden.

\begin{declaration}        %this creates the heading for the declaration page

I herewith declare that I have produced this paper without the prohibited assistance of third parties and without making use of aids other than those specified; notions taken over directly or indirectly from other sources have been identified as such. This paper has not previously been presented in identical or similar form to any other Irish or foreign examination board.

The Final Year Project was conducted from 2019 to 2020 under the supervision of James Patten at University of Limerick.

\vspace{10mm}

Limerick, 2020


\end{declaration}


% ----------------------------------------------------------------------


%: ----------------------- tie in front matter ------------------------

\frontmatter

% Thesis Acknowledgements ------------------------------------------------


%\begin{acknowledgementslong} %uncommenting this line, gives a different acknowledgements heading
\begin{acknowledgements}      %this creates the heading for the acknowlegments

First and foremost I would like to thank James Patten; both for his inspiration of the topic researched, and the constant helpful advice and motivation throughout. This extends to all of the lecturers and teaching assistants I have interacted with over the years, each of whom contributed to where I am today.

The wide breadth of open source code, lectures and tutorials available online were also essential in allowing me to carry out this project, as it was all new ground compared to schoolwork that came before it. 

\end{acknowledgements}
%\end{acknowledgmentslong}

% ------------------------------------------------------------------------



% Thesis Dedictation ---------------------------------------------------

\begin{dedication} %this creates the heading for the dedication page

\vspace{31mm}

Shoutout Christina Applegate

\end{dedication}

% ----------------------------------------------------------------------


%: ----------------------- contents ------------------------

\setcounter{secnumdepth}{3} % organisational level that receives a numbers
\setcounter{tocdepth}{3}    % print table of contents for level 3
\tableofcontents            % print the table of contents
% levels are: 0 - chapter, 1 - section, 2 - subsection, 3 - subsection


%: ----------------------- list of figures/tables ------------------------

\listoftables  % print list of tables
\listoffigures	% print list of figures
\lstlistoflistings % print list of code snippets
\addcontentsline{toc}{chapter}{Listings: List of Code Snippets}

%: ----------------------- glossary ------------------------

% Tie in external source file for definitions: /0_frontmatter/glossary.tex
% Glossary entries can also be defined in the main text. See glossary.tex
%\include{0_frontmatter/glossary} 
%
%\begin{multicols}{2} % \begin{multicols}{#columns}[header text][space]
%\begin{footnotesize} % scriptsize(7) < footnotesize(8) < small (9) < normal (10)
%
%\printnomenclature[1.5cm] % [] = distance between entry and description
%\label{nom} % target name for links to glossary
%
%\end{footnotesize}
%\end{multicols}



%: --------------------------------------------------------------
%:                  MAIN DOCUMENT SECTION
% --------------------------------------------------------------

% the main text starts here with the introduction, 1st chapter,...
\mainmatter

\renewcommand{\chaptername}{} % uncomment to print only "1" not "Chapter 1"

\lstset
{ 
    language=Python,
    basicstyle=\ttm,
    otherkeywords={self},             % Add keywords here
    keywordstyle=\ttb\color{deepblue},
    emph={MyClass,__init__},          % Custom highlighting 
    emphstyle=\ttb\color{deepred},    % Custom highlighting style
    stringstyle=\color{deepgreen},
    numbers=left,
    stepnumber=1,
    showstringspaces=false,
    tabsize=1,
    breaklines=true,
    breakatwhitespace=false,
    captionpos=b
}

%: ----------------------- subdocuments ------------------------

% Parts of the thesis are included below. Rename the files as required.
% But take care that the paths match. You can also change the order of appearance by moving the include commands.


% this file is called up by thesis.tex
% content in this file will be fed into the main document

%: ----------------------- introduction file header -----------------------
\chapter{Introduction} \label{introduction}

% the code below specifies where the figures are stored
\ifpdf
    \graphicspath{{1_introduction/figures/PNG/}{1_introduction/figures/PDF/}{1_introduction/figures/}}
\else
    \graphicspath{{1_introduction/figures/EPS/}{1_introduction/figures/}}
\fi

% ----------------------------------------------------------------------
%: ----------------------- introduction content ----------------------- 
% ----------------------------------------------------------------------

The overall goal of this report is to investigate what makes a good story, and then investigate methods for algorithmically generating them. The question of "Why?" may be asked. What makes stories important enough to warrant this kind of work?

Stories are widely believed to be incredibly useful for fostering an understanding of the shared human experience and questions of existence (\Citealt{eder2010life}), for educational communication (\Citealt{birch1996says}) and at their most incisive, contributing to social and political change (\Citealt{fuertes2012storytelling}). This is all in addition to the entertainment value we all gain from stories, whether they be written, recorded, or shared via word of mouth. These stories develop whole ideologies and cultures; one need only look at religions for evidence, and we have records of written literature dating back to 2600 BCE (\Citealt{grimbly2013encyclopedia}).

For these reasons, we believe this research to be incredibly worthwhile. If we can aid or expedite the writing process with the introduction of procedurally generated contributions, this has the potential to serve as a tool to foster even more creativity in human writers. Inspiration for this is drawn from Google Deep Mind (\href{https://deepmind.com/}{https://deepmind.com/}) and their Alpha Go (\Citealt{silver2017mastering}) research: an artificial intelligence (AI) program designed to tackle the ancient Chinese board game of Go (\href{https://www.cs.cmu.edu/~wjh/go/rules/Chinese.html}{https://www.cs.cmu.edu/~wjh/go/rules/Chinese.html}). While much of the detail of the research they conducted is beyond the scope of ours, one takeaway from their research was that humans learned and became better for having played the AI program, in the same way they might improve while playing a superior human. It is hoped that the application developed as a result of this research could serve in a similar manner, seeing improvements in human writers with the help of an AI writer, as well as the AI's productions themselves.

\vspace{10mm}

% here we declare a new section

\section{Aims and Objectives} 

Two main questions to be addressed are: what is a good story, and how may they be procedurally generated? In order to address these questions, this work focuses on the following areas:

\begin{itemize}

\item An examination of stories from a linguistic perspective. Older research or that which does not concern itself so much with technology. This includes things like story structures, character development, plot devices and so on. Creating a perfect definition for a "good story" may be implausible, but each strand and structural element that we can understand and incorporate gets us closer to understanding the nature of stories.

\item Algorithms for formulating these elements in such a way as to generate stories. We will look at a number of candidates, evaluate each of them their general merits and then decide which one shows the greatest potential. 

\item Technologies that utilise these algorithms, researching advancements that have been made from early stage implementations up through state-of-the-art. This will be done with the product in mind, choosing which way to approach its creation.

\item From here we will produce a prototype product which will generate stories and allow human users to interact and modify these stories as they go. In the same way that a human-computer combination has proved more effective than a human alone in chess (\Citealt{michie1972programmer}), we aim to create an interactive co-writing experience.

\end{itemize}



% here we declare a new section
\section{Methodology} 

In order to address these questions we will follow the methodology outlined in sequence above, before designing the system itself and implementing the product. Afterwards, we will reflect and perform various types of evaluation on the productions of the system, noting that it is not primarily intended to be completely autonomous but rather with human interaction. Evaluation of the output will include known language generation metrics as well as human review.
 
 
% here we declare a new section
\section{Research Contribution}

A comprehensive history and discussion of what makes stories a worthwhile endeavour; how their quality and elements may be defined; and ultimately  a prototype that demonstrates the potential of automated techniques to act as an aid for the human creative in the story writing process.

\section{Report Outline} 

The remaining chapters of this report are as follows:


\emph{Chapter Two} outlines and discusses the history and related research to this topic, from linguistic and technological perspectives; \emph{Chapter Three} relates to the design of my chosen system and the choices that were made with regard to models, architecture etc; \emph{Chapter Four} presents the implementation of the system, portions of interesting code, struggles that were faced and how I overcame them; \emph{Chapter Five} deals with the evaluation of productions from the system, both objective metrics and subjective human review; \emph{Chapter Six} draws conclusions, evaluates my satisfaction and areas for improvement, and suggests possible future works and research.

%\vspace{5 mm}

%A series of documents have been included in the Appendix section of this dissertation. These are:
%
%\begin{itemize}
%\item \emph{Appendix A} outlines . . .
%
%\item \emph{Appendix B} presents . . .
%
%\item \emph{Appendix C} includes . . .
%\end{itemize} 
%
%\vspace{5 mm}
%
%Attached to this dissertation is a CD containing the following items:
%
%\begin{itemize}
%\item \emph{folder 1}: . . .
%
%\item \emph{folder 2}: . . .
%
%\end{itemize}
	
% this file is called up by thesis.tex
% content in this file will be fed into the main document

%: ----------------------- name of chapter  -------------------------

\chapter{Related Research} % top level followed by section, subsection

%: ----------------------- paths to graphics ------------------------

% change according to folder and file names
\ifpdf
    \graphicspath{{2_related_research/figures/PNG/}{2_related_research/figures/PDF/}{2_related_research/figures/}}
\else
    \graphicspath{{2_related_research/figures/EPS/}{2_related_research/figures/}}
\fi

%: ----------------------- contents from here ------------------------

Much research has been undertaken on this topic, with a variety of approaches.


\section{Stories}

It is important for us to understand the most basic, linguistic concepts before we move on to algorithmic productions. What are the elements that make up a good story? What constitutes a good story? What even is a story? 

According to Webster (\Citealt{dictionary2002merriam}), a \emph{Story} is: 
\begin{quote}
An account of incidents or events, [either] regarding the facts pertinent to a situation in question, [or] a fictional narrative.
\end{quote}

With a \emph{narrative} being:
\begin{quote}
A way of presenting or understanding a situation or series of events that reflects and promotes a particular point of view or set of values.
\end{quote}

This is naturally a broad set of definitions, but it does give us some cues. From a story, we would expect a series of events related to each other in some way, either to describe a situation or promote a certain worldview or message. That is, there is an overall connection and cohesiveness to the piece. Literary critics have posited several interpretations or generalisations: that stories begin in equilibrium before being disrupted, and ultimately involve a journey back to equilibrium (\Citealt{todorov1969structural}), but yet more argue that there can be no "correct" definition determined (\Citealt{sullivan2002reception}). The fields of Literary Theory and Narratology emerged in an attempt to dissect and formulate stories, which we will discuss more later.

I first set out to examine commonly used techniques used in various aspects of storytelling. These will perhaps be more relevant in evaluating the productions of my system, rather than guiding them too much, depending on the level of autonomy, and the rigidity of training and generation it has.

\subsection{Structure}

Structure describes the underlying framework of a story and, as the highest level of a story planning process, was first to be investigated. 

The classic structure would be the three acts. This dates back to Aristotle in 400BC, describing a story as having three parts: a beginning, an end and a middle (\Citealt{mack1980norton}), and is still popular today. It is commonly depicted something like (Figure \ref{3-act-structure}).
 
\figuremacroW{3-act-structure}{Three Act Structure}{Breakdown of three act structure. [Source: (\Citealt{3-act-structure-image})]}{0.8}

Stories are broken town into distinct sections: setup and exposition, rising action and confrontation, climax and resolution (\Citealt{trottier1998screenwriter}). This is a bit more versatile, which is something to consider for the later steps of “filling in the gaps” with our algorithmic approach. A balance must be struck between: providing some structure so that the generation has some level of cogency, but also allowing flexibility so that not all stories are the same.

Other popular methods include the Hero’s Journey (\Citealt{campbell2008hero}), which describes a cycle of sorts: a call to adventure, crossing from the known to the unknown, transformation etc. This is a more granular structure, most known for its application in Star Wars. See (Figure \ref{heros-journey}).
 
\figuremacroW{heros-journey}{The Hero's Journey}{A cycle of the hero's journey. [Source: (\Citealt{vogler1985practical})]}{0.8}

However, that is not to say these are the only structures that must be followed, nor that there must be that traditional arc. Some authors have examined spiral, fractal and explosive patterns in literature (\Citealt{alison2019meander}), rejecting the historical, structural norms. It is tempting to declare adherence to a structure irrelevant, but patterns do remain, so this cannot be ignored entirely.

\subsection{Plot}
\label{story_plot}

Delving deeper into the story, plot makes up those events which are significant, have consequences and make a difference to the story (\Citealt{dibell1999elements}). If we examine our three act structure figure from earlier (Figure \ref{3-act-structure}), we see ticks along the line of the story, larger incidents which have an impact.

Indeed, a scene or series of events may be memorable and iconic, but if they do not serve as major events that progress the overall narrative, then they do not constitute plot (\Citealt{alcorn2014know}). Following on from the three-act structure, plot points would be used to connect the acts to each other, for example: our protagonist is thrust into an unexpected situation, they face a setback and it seems all hope is lost, finally they overcome.

For the context of our system, plot points should serve as transitional pieces, advancing the story in some way so we don't simply remain at or revert back to the previous content. This must be handled with care, as too few plot points would be boring, but too many would be bewildering.

\subsection{Setting}

This refers to the time, location and milieu in which the story occurs (\Citealt{lodge2012art}) often referred to as the "world" or "universe", in modern works. 

This serves as the backdrop of our story, and to feel authentic it must be rich with context and history. At their best, settings are so specific that they provide natural associations to the reader (\Citealt{kuntz1993narrative}), setting the mood and plot anticipations.

When generating content, there should be at minimum a consistency of setting, and ideally it should have enough detail to establish a mood that carries through the story.


\section{Narratology}

Having touched on literary theories, I was set on to the work of Vladimir Propp on Narratology (the study of narrative) and his early work in formulating elements of stories. Specifically, his seminal work with Morphology of the Folktale (\Citealt{propp1968morphology}), originally written in 1928. He, along with other Russian formalists, took a modern approach to narratology after Aristotle's ancient theorising.

They distinguished the syuzhet (plot) from the fabula (story). The idea was that the story is the raw material, familiar in many ways already, and it is \emph{defamiliarised} (a term they coined) into the plot, a new organisation and the way the story is told. This goes back to our previous point on Plot (\ref{story_plot}), which pointed out that plot pertains to the information that pushes a story along. They are subtly different concepts.

\subsection{Abstractions}

Propp's work is very relevant to this research, since he has some of the earliest work on abstracting and formulating aspects of stories (specifically Russian folktales).

He first curated a table of possible events at various stages of a story, then associated these with symbols and combined them with functions into what look like mathematical formulas. An example looks something like: (Figure \ref{propp-eg}).
 
\figuremacroW{propp-eg}{Morphology of the Folktale}{Breaking down a story. [Source: (\Citealt{propp1968morphology})]}{0.8}

There were hundreds of these, assembled in a tabular format like so: (Figure \ref{propp-eg-2}).
 
\figuremacroW{propp-eg-2}{Morphology of the Folktale}{Table of stories. [Source: (\Citealt{propp1968morphology})]}{0.8}

This laid groundwork for the development of grammars, while this is perhaps too rigid and systemic to be applicable, as argued by some (\Citealt{dundes1997binary}). 


\section{Formal Language Theory}

(Formal) Grammars were devised as a more generic and granular approach to generating strings of text, by following certain rules, from a certain alphabet (\Citealt{reghizzi2013formal}). Compared to Propp's work on formulating \emph{elements}, this was work being done down to the character level, getting closer to what we would need for algorithmic generation.

Emil Post was one of the early innovators in this area, creating the Post Canonical System in 1943, a string manipulation system for generating instances of a language, from an initial alphabet and rules (\Citealt{post1943formal}). 

\subsection{Grammars}

Noam Chomsky then proposed a set of generative grammars in 1956, classified in the \emph{Chomsky Hierarchy} (\Citealt{chomsky1956three}), with different levels of strictness in their rules. The two efficient and popular types were the Context Free Grammar and Regular Grammar.

Chomsky grammars consist of a finite set of production rules (left-hand side$\,\to\,$right-hand side), where each side consists of a finite sequence of the following symbols:
\begin{itemize}
\item a finite set of nonterminal symbols (indicating a production rule can be applied)
\item a finite set of terminal symbols (indicating no production rule can be applied)
\item a start symbol (a distinguished nonterminal symbol that is not found on any right hand side, and so cannot be produced anyway else)
\end{itemize}

For example:

\begin{equation}
    S \rightarrow AB
\end{equation}\begin{equation}
    S \rightarrow \lambda (empty string)
\end{equation}\begin{equation}
    A \rightarrow aS
\end{equation}\begin{equation}
    B \rightarrow b
\end{equation}

This is a Context Free Grammar (CFG) that could generate a string of letters "a" and "b". 

Computing systems for grammar and story generation have been built (\Citealt{compton2014tracery}, )however, formal grammars like this require significant human building and labelling. While these are interesting and worth exploring from a historical perspective, they are troubling from the perspectives of extensibility and originality.


\section{Natural Language Generation}

A subfield of linguistics and computer science, Natural Language Processing (NLP) deals with making easier the interaction between humans and computers, enabling machines to more easily understand natural, human language. It has risen to prominence since the 1990s in line with the rise of machine learning as a programming technique (\Citealt{johnson2009statistical}). Before this, early language processing systems were based on handwritten rules like the grammars outlined above (\Citealt{schank2013scripts}), which meant a comparative lack of knowledge, or features – weights on different choices based on the data. This machine learning approach allowed programs to learn rules by themselves, via analysing large amounts of input data.

Further developments in the 2010s brought the advancement of deep learning and neural networks, which could achieve better results than ever before (\Citealt{goldberg2016primer}). We will examine these in more detail shortly, as these results made them quickly became the leading contender for our system.

It is worth noting that the NLP field is wide, and for our purposes we will primarily focus on the area of Natural Language Generation (NLG). Natural Language Understanding, which we are perhaps more familiar with in things like virtual assistants, aims to take in natural language, abstract it and produce some kind of representation of the idea being conveyed (\Citealt{reiter2000building}). Conversely, NLG attempts to take what is structured data, weights and biases based on input data, and produce natural language.

\subsection{Neural Networks}

An Artificial Neural Network (ANN) is a computing system that seeks to emulate the kind of processing and problem solving done by the human brain, designed around pattern recognition (\Citealt{haykin1994neural}). The core principle is that they “learn” to perform said task by analysing examples and figuring out their own rules, rather than being explicitly told any rules up front. 

An ANN is modelled after the human brain, containing nodes, or "neurons", which are connected to each other. They can receive input, process this with their internal state, and produce output (\Citealt{winston1992artificial}). This can be passed as a message to another neuron or ultimately, complete the task at hand. A basic example of the architecture looks something like the following: (Figure \ref{nn-sample}).
 
\figuremacroW{nn-sample}{Artificial Neural Network Architecture}{Neurons and connections. [Source: (\Citealt{nn-sample-img})]}{0.6}

Each connection between neurons has an associated weight, which represents its relative importance, which is taken into account by the receiving neuron processing its inputs.

The classic example and popular use case is image identification (\Citealt{le2013building}). To teach a program to correctly separate images of cats and dogs, the basic approach would be to give it explicit rules for cats versus dogs – nose, ears, paws, etc. However, the ANN allows you to simply feed it examples (ideally many) of images of both, and then allow it to formulate those rules by itself. This saves time and work for the user, and tends to create a much more accurate model (\Citealt{lecun1989backpropagation}), especially with a large amount of input, "training" data.


\subsubsection{Recurrent Neural Networks \& LSTMs}

Examining some of the recent works on natural language generation and even specific story generation papers, Recurrent Neural Networks (RNNs) stood out as the popular technology (\Citealt{peng2018towards}), (\Citealt{fan2018hierarchical}), (\Citealt{sutskever2014sequence}). 

These are like neural networks, but with loops that provide a kind of memory or persistence (\Citealt{graves2008novel}). These have feedback connections and can process sequences of data, instead of just single data points like a typical (feedforward) neural network. Particularly Long Short Term Memory (LSTM) networks, which are better at remembering long term dependencies. See (Figure \ref{rnn-unrolled}).
 
\figuremacroW{rnn-unrolled}{Recurrent Neuron}{Loops for persistence [Source: (\Citealt{olah2015rnn})]}{0.8}

This is incredibly important for NLG, because context matters. When identifying images, as with the previous example, this isn’t such a big deal. Examples are distinct and don’t depend on each other – one input, one output. However, with video or language, you must generate the next part of your output, piece by piece. In this case, remembering what was written previously is vital, not only to make coherent sentences but also to construct an overall narrative throughout the story. Characters should develop, plot points should advance, relevant twists should occur, and so on.


\subsection{Transformers}

However, these RNNs and LSTMs appear not to be state of the art in the NLP world anymore. Things move quickly, and that has led us to the Transformer architecture. 

The idea was pioneered by a team of Google researchers (\Citealt{vaswani2017attention}) and the early results are very promising. OpenAI’s GPT-2 (\Citealt{radford2019language}) is perhaps the most advanced language processing Artificial Intelligence (AI) system currently, so much so that they declined to release their full code, claiming they fear it could be used for ill means e.g. fake news.

We mentioned RNNs utilising loops, but in reality it’s more like a series of connected Neurons, each performing its processing and then passing the new “state” to the next. This is what allows it to maintain memory – each stage of the computation is aware of everything that happened previously. You can see this in the following, with the RNN on top and Transformer on bottom: (Figure \ref{attention}).
 
\figuremacroW{attention}{RNN vs. Transformer Architecture}{Attention replacing loops [Source: (\Citealt{kurita2017attention})]}{0.8}

Note: here the example is translation, but for language generation the idea is the same. Instead of passing in a passage and translating it, we seek to find the next part of said passage. Note also that there are encoding (orange) and decoding (blue) stages, but they are practically similar. 

The issues here are clear. The RNN is very difficult to parallelise since the computation must be done in sequence, passing the state along (\Citealt{bengio1994learning}). Also, the further you drift from the start (i.e. the longer the text becomes) the less your network will remember about those early stages. This is a problem for stories, where elements are often mentioned briefly at the start, only to come into prominence later – vital for continuity.

Transformers, by contrast, take the entire input passage in at once, using the concept of Attention (\Citealt{vaswani2017attention}) along with positional encoding (to ensure you don’t start mixing up a sentence) to calculate the relative importance of each word. These results are then passed through a more traditional feedforward Neural Network (much faster than an RNN) and potentially another multi-head attention stage (applications differ) to produce your following piece of the passage. 

For the purposes of this research, we won’t be delving into the inner workings of the Attention mechanism, but essentially what it does is calculate the relevance of each word in the input. For example, if you have the input “I grew up in France… I speak fluent” and wish to predict the next word, the word “France” will be given a much higher score, indicating to the system what it should predict based on.

This helps hugely with the long term dependency issue, since at each stage the entire input is considered, and thus there is a much lower risk of earlier information being “forgotten”, something that even the finest LSTMs struggled with. 

This could lead to performance issues, but thanks to the Attention mechanism and feedforward NN being used (allowing you to process the entire input at once), Transformers are in fact faster.

The best models also seem to utilise unsupervised learning (to be discussed later), which has the obvious benefit of being able to feed it much more data, as seen with GPT-2 and its 1.5 billion parameters (\Citealt{radford2019language}). However, you do need to feed it a massive amount of data for it to be in any way effective. 


\subsubsection{BERT}

Google

\subsubsection{GPT}

First attempt

\subsubsection{GPT-2}

New and improved!



% ---------------------------------------------------------------------------
%: ----------------------- end of thesis sub-document ------------------------
% ---------------------------------------------------------------------------


% this file is called up by thesis.tex
% content in this file will be fed into the main document

\chapter{Design} % top level followed by section, subsection

\ifpdf
    \graphicspath{{3_design/figures/PNG/}{3_design/figures/PDF/}{3_design/figures/}}
\else
    \graphicspath{{3_design/figures/EPS/}{3_design/figures/}}
\fi

% ----------------------- contents from here ------------------------




\section{Language Model}

We identified a range of potential models for this project, ranging from earlier grammars, to more recent Recurrent Neural Networks, and finally Transformers. For demonstration purposes, we decided it would be best to settle on one model.

\subsection{GPT-2}

As discussed in Section~\ref{gpt}, GPT-2 is currently the state-of-the-art model when it comes to the generation of natural language. A number of resources were identified which made it easier to utilise and fine-tune this model \footnote{\href{https://talktotransformer.com/}{Talk to Transformer} (https://talktotransformer.com/) by Adam King} \footnote{\href{https://github.com/minimaxir/gpt-2-simple}{GPT-2 Simple} (https://github.com/minimaxir/gpt-2-simple) by Max Woolf}, as well as the original open-source code itself \footnote{\href{https://github.com/openai/gpt-2/}{GPT-2} (https://github.com/openai/gpt-2/) by OpenAI}. 

This made the task somewhat less complex, but our original intention at the time was to modify the underlying architecture in some way. Training a model, we reasoned, was too basic, and we needed something more. Since the code is open source, we set about reading through it, along with many explainer articles \footnote{\href{http://jalammar.github.io/illustrated-gpt2/}{The Illustrated GPT-2} (http://jalammar.github.io/illustrated-gpt2/) by Jay Alammar} and videos \footnote{\href{https://youtu.be/S0KakHcj_rs}{Attention Is All You Need} (https://youtu.be/S0KakHcj\_rs/) by the AI Socratic Circles, was helpful in understanding the Attention mechanism}.

However, the enormity of this challenge was soon realised. These were concepts and models developed by teams of post-doctorate level researchers, working full-time on these problems, with the backing of huge corporations. For myself, just getting started in the field of machine learning and NLP, it didn't seem like the most impactful use of my time, or the most value I could contribute through my research. 

We took a step back, and considered where I might add value to the system. We discussed the fact that these models, despite being significant advancements and impressive research, were still nowhere near consistently challenging a human writer, and that curation of the AI productions was still required. This seemed like an area for exploration, to create an environment for human writers to work together with a language generation model, editing and inserting their own text as they go. A decision was made to pursue this direction.

\subsection{Implementation}

The implementation of GPT-2 I settled on was \href{https://huggingface.co/}{Huggingface}'s (https://huggingface.co/) repository of \href{https://huggingface.co/transformers/}{Transformers} (https://huggingface.co/transformers/). This contains not only GPT-2, but a collection of all state-of-the-art architectures, which leaves room for extensibility and including other models in the future.

The code provided in this repository allows for a variety of approaches for optimal accessibility: from simply running generation on a pre-trained model, to fine-tuning your own model, to custom generation or even altering the lower level code. A variety of scripts were included, making it quick and straightforward to get up and running. This seemed ideal for our purposes.

\subsection{Training}

One concern that had prevailed since we delved into NNs was the hardware that is required to train these models. Transformers utilise Convolutional Neural Networks, which are faster than the Recurrent variant but still require significant computing power. Throughout the development process there was limited availability of, and accessibility to, hardware, which are especially important for the training phase.

Fortunately, alternatives do exist, and there are numerous cloud services available for training machine learning models on third-party hardware: \href{https://www.kaggle.com/}{Kaggle}, \href{https://lambdalabs.com/}{Lambda} and \href{https://www.wandb.com/}{Weights \& Biases}, just to name a few. There were drawbacks, namely cost, but these were promising.

Having analysed all the available alternatives a decision was made to proceed with \href{https://colab.research.google.com/}{Google Colab} (https://colab.research.google.com/), which provides a Python Notebook interface and allows you to run code on advanced (NVIDIA Tesla K80) GPUs. There are limits on time and use cases, but I was comfortable I could fall within or work around these restrictions, especially considering that the product was free to use. Of course, if we wanted to train a model then we would need data.

\subsection{Data}

Another issue that came up in the early stages was scope. Often an issue with projects of this nature, we wanted to make sure that we had a reasonable prototype which demonstrated progress. To generalise from the beginning would make the model's ability to optimise less clear, so we decided to focus on one type of story initially and perhaps generalise later, or at least leave that option open to ourselves.

I was careful when choosing a genre, since it was important to have a large and high quality data source to support it. Being a \href{https://www.reddit.com/}{Reddit} (https://www.reddit.com/) user for some time, I recognised the wealth of data available on the website, not just as a whole but categorised neatly into different sub-communities dedicated to certain topics. 

The \href{https://www.reddit.com/r/writingprompts}{Writing Prompts} (https://www.reddit.com/r/writingprompts) subreddit was the leading contender. A forum devoted to short story writing, it seemed like the perfect match. However, it was not specific to any genre of story and categorisation of the stories was minimal, meaning it was more of a general data source. Good, but not quite what we needed.

having carefully analysed the alternatives, a decision was made to use \href{https://www.reddit.com/r/nosleep}{No Sleep} (https://www.reddit.com/r/nosleep), a community for short horror stories which, as the title goes, denied their users sleep. This was much more suitable, with focused data and a significant amount of it too. As of 26 March 2020, the subreddit has been going for 10 years with 13.9 million subscribers. This was an ideal data source for our purposes.


\section{User Experience}

Simple webapp, user interaction, generate and change parameters, add/edit text as you go (Sketches of UI and structure)

\subsection{Back End}

Python/Flask API to call scripts based on huggingface implementation, take in parameters and text, return json with text object of generated string to replace existing one

\subsection{Front End}

ReactJS framework, library for building UI. Make calls to API and display results.



%\subsection{Creating a Table}
%
%\begin{table}[htp]
%\caption{Table Title}
%\begin{center}
%\begin{tabular}{| p{3cm} | c | c | c |}
%\hline 
% & Resolution & Min & Max \\ \hline 
%Gyroscope & 4.5mV / {$^{\circ}$}/s & 0.27 $^{\circ}$/s & 406 $^{\circ}$/s \\ \hline
%Accelerometer & 600mV /g &  0.002g & 2g \\ \hline
%Magnetometer & 385mV/ gauss & 0.317 gauss & 6 gauss \\ \hline
%\end{tabular}
%\end{center}
%\label{Sensors' Resolution}
%\end{table}%


%% here is how to do a page break
%\pagebreak
%%done
%
%\section{Quote}
% An in-text quote is declared in the following way:
%
%\begin{quote}
%
%Recycling is the way forward! Recycling is the way forward! Recycling is the way forward! Recycling is the way forward! Recycling is the way forward! Recycling is the way forward! Recycling is the way forward! Recycling is the way forward! Recycling is the way forward! Recycling is the way forward! Recycling is the way forward! Recycling is the way forward! Recycling is the way forward! Recycling is the way forward! Recycling is the way forward! Recycling is the way forward! Recycling is the way forward! Recycling is the way forward! Recycling is the way forward!    (CommonSense, p. 0) 
%
% \end{quote}
%
%Another way of doing it is:
%\begin{quote}
%\centering
%\emph{There are in our existence spots of time, \\
%That with distinct pre-eminence retain \\
%A renovating virtue, whence-depressed \\
%By false opinion and contentious thought, \\
%Or aught of heavier or more deadly weight, \\
%In trivial occupations, and the round \\
%Of ordinary intercourse-our minds \\ 
%Are nourished and invisibly repaired; \\ 
%A virtue, by which pleasure is enhanced, \\ 
%That penetrates, enables us to mount, \\ 
%When high, more high, and lifts us up when fallen. \\ 
%}
%\vspace{3mm}
%\raggedleft (\Citealt[verses 208-218]{poem})
%\end{quote}


%\section{Creating a list}
%
%\begin{itemize} \itemsep1pt \parskip0pt \parsep0pt
%\item \textbf{(2000)} item 1. 
%\item \textbf{(2004)} item 2. 
%\item \textbf{(2010)} item 3. 
%\item \textbf{(2013)} item 4. 
%\end{itemize}





% ---------------------------------------------------------------------------
% ----------------------- end of thesis sub-document ------------------------
% ---------------------------------------------------------------------------		
% this file is called up by thesis.tex
% content in this file will be fed into the main document

\chapter{Implementation} % top level followed by section, subsection


% ----------------------- paths to graphics ------------------------

% change according to folder and file names
\ifpdf
    \graphicspath{{4_implementation/figures/PNG/}{4_implementation/figures/PDF/}{4_implementation/figures/}}
\else
    \graphicspath{{4_implementation/figures/EPS/}{4_implementation/figures/}}
\fi


% ----------------------- contents from here ------------------------

\section{Web Scraping}

Gathered training data from reddit, few different approaches, different sized datasets produced.

As outlined in Section~\ref{data}, we decided to utilise the \href{https://www.reddit.com/r/nosleep}{No Sleep} (https://www.reddit.com/r/nosleep) community on reddit as a data source, owing to its focused nature on horror stories and mass of posts. Reddit does offer API support itself, so naturally this was the first route we took.

\subsection{Reddit API and PRAW}

Limited to 1000 posts

The \href{https://www.reddit.com/dev/api/}{Reddit API} (https://www.reddit.com/dev/api/) is freely available to developers, and offers plenty of functionality and customisation, not limited to: date ranges, filtering based on points, removing posts by certain users, etc. It perhaps isn't the easiest to use, however, but there also happen to be many API wrappers available – effectively providing a layer of abstraction and a more user-friendly interface over the core API.

\href{https://praw.readthedocs.io/en/latest/}{PRAW} (https://praw.readthedocs.io/en/latest/) (Python Reddit API Wrapper) provides, as the name suggests, a Python interface, together with plenty of documentation and examples. we initially ran scripts like Listing~\ref{prawscrape}, which would scan through the top 1000 posts in a subreddit, add the "post" objects into a list, and then iterate over that list to add all of the posts' content into a text file. 

\lstinputlisting[fontadjust=\true, float, firstline=16,lastline=26, caption=PRAW Scrape Script, label={prawscrape}]
{../../../../code/data/scripts/praw_scrape.py} 

However, we soon ran into a problem, which was that the limit on post requests at a time was indeed 1000. This meant that going through in one call was going to be impossible or very messy with this API approach.

\subsection{PushShift API}

\href{https://pushshift.io/}{PushShift} (https://pushshift.io/) is a third-party data source which collects and collates reddit data into its own databse. It also provides an API, allowing us to query that with no artificial post limit. This was done with a more traditional http request as seen in Listing~\ref{pushscrape}. PushShift was slower but unlimited, meaning we could simply loop through pages of results (starting with the most recent), keeping track of the last post's creation date (line 23) and make sure that our next iteration precluded posts after that date. This took time but allowed us to assemble a variety of datasets, as discussed in Section~\ref{datasets}.

\lstinputlisting[fontadjust=\true, float, firstline=10, caption=PushShift Scrape Script, label={pushscrape}]
{../../../../code/data/scripts/pushshift_scrape.py} 


\subsection{Datasets} \label{datasets}

As discussed in Section~\ref{data}, we wished only to collect posts with a relatively high score as voted by the users of the community, and several datasets were assembled on a trial-and-error basis.

\begin{itemize}

  \item Initially, with the official Reddit API, a file with the top 1000 posts was created, a size of 12MB.
  \item Using PushShift, we expanded our scope and collated all posts with over 1000 points, for a size of 45.4MB.
  \item Feeling this wasn't enough, the points threshold was reduced to 400, which gave us a file size of 102.5MB.
  \item As a stretch goal, we set a very low bound on points, only 10 points, which created a file of 496.3MB.

\end{itemize}

However, when training these models individually for testing purposes, we found that perplexity (a measure for evalutating text generation, which will be discussed more in Section~\ref{evaluation}) actually worsened when moving from the first to the third dataset. As such, we decided to stop training and proceed with the third model, despite its slightly worse score, fearing the exponential increase in size from the fourth dataset.


\section{Training the Model}

Worried about hardware, online resources to the rescue! Sample scripts for GPT-2 provided by huggingface

The training process for these models revolved around Google Colab, as designed in Section~\ref{training}. Here, we were able to connect the Python Notebook environment to Google Drive storage, where we uploaded the relevant scripts for training as well as the datasets from which we were building the models. The environment looked something like Figure~\ref{colab_env}. We can see the file explorer on the left-hand-side, which allows access to Google Drive files via the /gdrive/ folder, and on the right we see the Notebook itself. Along the top toolbar we have menus including Runtime, which is where we connect to a remote hardware accelerator (Nvidia K80 GPU).

\figuremacroW{colab_env}{Google Colab Environment}{Python Notebook}{1}

Initial setup code involved mounting Google Drive and importing various required packages, as in Listing~\ref{colab_setup}. As you can see, we were able to run scripts and typical unix terminal commands using the (!) prefix.

\lstinputlisting[fontadjust=\true, float, firstline=12, lastline=22, caption=Colab Setup, label={colab_setup}]
{../../../../code/data/scripts/hf_transformers_train.py} 

Then, we utilised the run\_language\_modeling.py script, provided as part of the Huggingface implementation (discussed in Section~\ref{modelImpl}), to carry out the main training process. This took a variety of parameters, notably: the input files for training and evaluation, the number of training epochs to perform, how often to save checkpoints and whether those should be resumed, and finally the output directory. A sample of how we used that script is in Listing~\ref{colab_train}. This "output" foler with our model could then be copied back over to Google Drive and downloaded.

\lstinputlisting[fontadjust=\true, float, firstline=28, lastline=42, caption=Colab Setup, label={colab_train}]
{../../../../code/data/scripts/hf_transformers_train.py} 

Checkpoints were stored at steps specified in the script, into folders in the same file system. These were very useful in resuming previous training progress, since it did take several hours at a time and Colab imposes a 12 hour time consecutive runtime limit, with a cooldown period between.


\section{React JS}

Web framework, modern JavaScript library for building UI.

The front end of our application was written in JavaScript, specifically using the React library we mentioned in Section~\ref{frontend}. From a software design point of view, we went for a purely functional style with no classes, as these have become less relevant in newer versions of React. We felt this would aid the development process by virtue of being highly modular and adaptable.

Code etc etc.

\subsection{Hooks}

Functional approach, no classes, using state which is passed around. Challenging new way of thinking but extensible and clean.


\section{Python Flask API}

API to encapsulate scripts, easily callable and customisable.

The Python Flask API (outlined in Section~\ref{backend}) would utilise.



% ---------------------------------------------------------------------------
% ----------------------- end of thesis sub-document ------------------------
% ---------------------------------------------------------------------------
	
% this file is called up by thesis.tex
% content in this file will be fed into the main document



%: ----------------------- name of chapter  -------------------------
\chapter{Evaluation} % top level followed by section, subsection

\ifpdf
    \graphicspath{{5_chapter5/figures/PNG/}{5_chapter5/figures/PDF/}{5_chapter5/figures/}}
\else
    \graphicspath{{5_chapter5/figures/EPS/}{5_chapter5/figures/}}
\fi

Evaluation of the system's productions was difficult for a number of reasons. Metrics available for language generation are limited, largely focused on classification (GLUE (\url{https://gluebenchmark.com}), question answering (SQuAD (\url{https://rajpurkar.github.io/SQuAD-explorer/}), or selection of the correct output from a set of options (ROCStories (\url{https://cs.rochester.edu/nlp/rocstories/}). These all relate to Natural Language Understanding, rather than Generation, and ultimately the gold standard in evaluating AI generated text is still human evaluation. 

Furthermore, since our system is intended to be used in collaboration with a human writer, pure metrics of the generation in isolation are not necessarily indicative of its efficacy. Human writers could easily correct small mistakes and vastly improve the quality of the story, but it is reasonable to suggest that we should still optimise the model to produce high quality and interesting text. It would be better to require less modification by the human.

\section{Perplexity}

We attempted to utilise some metrics, namely Perplexity (Section~\ref{perplexity}), at the very least to establish a baseline versus other models, particularly the non-fine-tuned GPT-2 from which we built our custom model on horror stories.

Perplexity is a straightforward measure for predicting how well a model can replicate validation data, from given training data (\Citealt{jelinek1977perplexity}). For our purposes, we split our initial datasets into training and validation sets (with an 80:20 split). In this case, a lower score is better, indicating the model is better at replicating stories. 

\begin{table}[ht]
\centering
\begin{tabular}[t]{lr}
\toprule
Model &Perplexity\\
\midrule
Top 1000 posts (\mytilde10MB)&19.4382\\
Posts over 400 points (\mytilde100MB)&21.1234\\
Default GPT2-small&37.50\footnotemark\\
\bottomrule
\end{tabular}
\caption{Perplexity Scores of Datasets and Control}
\label{perplexity_table}
\end{table}%
\footnotetext{Source: (\Citealt{radford2019language})}

As we can see in Table~\ref{perplexity_table}, both of our datasets produced models with far better perplexity than the default GPT-2 small model. This is to be expected, since it is an earlier version of GPT-2 and trained on a much more generic dataset. With a broader range of text, prediction is naturally more difficult. One concern was our larger dataset actually received a slightly worse perplexity score than the original, smaller set. This required some consideration.

We reasoned that taking a narrower slice of stories, the top 1000 posts of the NoSleep subreddit, they seemed to have more in common. Thus, prediction was easier, since the stories were more similar. This had interesting implications, suggesting that our original research into stories (Section~\ref{stories}) had correctly surmised that there are common elements to good stories. At least, among popular ones.

However, we felt that despite the marginal loss in complexity, the tenfold gain in data from which to draw on made the larger dataset the preferable option to proceed with. It was important for our system to have some degree of variety and creativity, even within the bounds of the horror genre, and having access to more data would aid this. We decided not to proceed further with larger datasets, owing to large computation costs and fear of tipping the balance, ending up with significantly worse complexity.


\section{Human Review}

More effective but obvious downsides in terms of speed. Part of the interactive experience.

\subsection{Self Review}

My judgement.

\subsection{Anonymous Reviews}

Posted various stories back to reddit to gauge response.


% ---------------------------------------------------------------------------
%: ----------------------- end of thesis sub-document ------------------------
% ---------------------------------------------------------------------------

	
% this file is called up by thesis.tex
% content in this file will be fed into the main document

\chapter{Conclusions and Future Directions} % top level followed by section, subsection


% ----------------------- paths to graphics ------------------------

% change according to folder and file names
\ifpdf
    \graphicspath{{6_conclusions/figures/PNG/}{6_conclusions/figures/PDF/}{6_conclusions/figures/}}
\else
    \graphicspath{{6_conclusions/figures/EPS/}{6_conclusions/figures/}}
\fi


% ----------------------- contents from here ------------------------


\section{Summary}

In the beginning, we set out to research the linguistic side of this topic, examining the essence of stories and the elements which make up good stories. This was intended to tie in with our research into procedural generation techniques and technologies, so that we could codify those elements to be utilised in algorithmic generation. Ultimately, the technology led us down a different path, culminating in the Transformer. Having grown much more complex and mathematical in nature, it was deemed that creating or modifying the core logic of this language generation technique was infeasible and unlikely to be the most significant contribution we could make. However, these models were still imperfect in their productions.

After much consideration, we proceeded to create an interface into the language model, with the intention of facilitating a collaborative writing environment between human and AI. The model itself was fine-tuned on a specific genre (horror), so as to provide proof of concept for the potential of Transformers to specialise in certain tasks, and subsets of tasks.

This research was an incredible learning experience for us, having to pivot at multiple points and reconsider the current path we were on. We regret that delving into the inner workings of a language model was not pursued, but remain satisfied with the product that was developed.  

\section{Contributions}

As outlined in Section~\ref{product}, we believe that there is significant potential for this product as well as collaborative AI in general. Procedural generation techniques, at least in the case of text, are clearly not yet at the level of advancement seen in more strict environments like board games, but they are sufficiently developed that they can help and inspire creativity in humans.

In many ways, the product born out of our research was more creative in nature, or an effort to facilitate creativity. We realised that attempting to conquer humans in this field was going to be infeasible for the forseeable future, but AI already has great ability to act as a writing partner of sorts. We believe that we have created the ideal live writing environment for this, and that there are many areas for expansion.


\section{Future Work}

Although we are happy with the progress of our prototype, there of course remains more work that could be done to improve it, including:

\begin{itemize}
\item \textbf{Titles:} These are currently manually entered, and having an element of procedural generation here would also be nice. Question answering were considered (\Citealt{devlin2018bert}), the idea being that it would be trained on stories as the question and titles as the answer, a model into wish we could pass our current story and receive a title "answer". This proved problematic, since the essence of question answering is uncreative, simply finding the correct answer. Thus, we wouldn't be generating any new or inventive titles, simply finding the existing one which matches best.
\item \textbf{Feedback:} The ability to provide feedback on the stories generated, or even passages throughout, would be helpful in terms of iterating on the model and fine-tuning it further. This also opens up the potential for customised user profiles, which would adapt the model to what the user has cited as good or bad output. With our training workflow on Google Colab, programatic re-training of the model was infeasible, but with sufficient hardware and running these processes locally, this could be possible.
\item \textbf{Model:} Returning to our early intentions, we still feel there is room for a more story-focused Transformer. Even since beginning this research, many different versions of Transformers have been released, but still none answering this more creative question. The gains would be incremental and perhaps not as extravagant, but there is so much more to learn and discover in this space.
\end{itemize}


% ---------------------------------------------------------------------------
% ----------------------- end of thesis sub-document ------------------------
% ---------------------------------------------------------------------------
             


% --------------------------------------------------------------
%:                  BACK MATTER: appendices, refs,..
% --------------------------------------------------------------

% the back matter: appendix and references close the thesis


% here we start the Appendix
\appendix

% Definition of each Chapter in the Appendix is done via the following two lines
\chapter*{Appendix A} \label{gui_screenshots}
\section*{GUI Screenshots}

% here is how we close the chapter
\addcontentsline{toc}{chapter}{Appendix A: GUI Screenshots}
\chaptermark{Appendix}
\markboth{Appendix}{Appendix}



\chapter*{Appendix B} \label{appendix:story}
\section*{AI Generated Story} 

\textbf{Once upon a time,} I even tried to leave the room, but nobody could help me. I knew that those buildings were haunted, and I knew that these people were worse than I had ever seen them. The only thing that kept me from going back was that I was haunted.

That evening, I knocked on the door and asked if anyone else wanted to come in. I told them I was a demon, and that I knew the answer to that. They all agreed, but their collective eyes were fixed on me.

This is what happened next.

I could hear the voices coming from downstairs. My bedroom. I could hear the sound of my heart beating. The sound of my mom crying. 

I called the police, but they couldn't do much. They told me they'd find out why I was there, but that it was too late.

I went back home and told them that I'd had a nightmare, but that I was going to go back down to the basement and try and sleep. My mom said that she didn't believe me, but I guess she wasn't too crazy. She said that if I didn't go back down to the basement, she'd know. She said that it was my fault. 

I didn't sleep. I didn't dream.

I couldn't forget. I didn't. I don't remember. There was the time that my mom found me in my room. I sat at the foot of my bed and tried to wake up. She told me that I had to go back to bed. She asked me to the stairs. I woke me to see how much that I was not happy, but she was wrong, that she didn't want to go back.

\addcontentsline{toc}{chapter}{Appendix B: AI Generated Story}
\chaptermark{Appendix}
\markboth{Appendix}{Appendix}

%





%: ----------------------- bibliography ------------------------

% The section below defines how references are listed and formatted
% The default below is 2 columns, small font, complete author names.
% Entries are also linked back to the page number in the text and to external URL if provided in the BibTex file.

% PhDbiblio-url2 = names small caps, title bold & hyperlinked, link to page 
%\begin{multicols}{1} % \begin{multicols}{ # columns}[ header text][ space]

\begin{small} % tiny(5) < scriptsize(7) < footnotesize(8) < small (9)

%\bibliographystyle{Latex/Classes/CUEDbiblio-url2} % Title is link if provided
%\bibliographystyle{plain}
%\bibliographystyle{Latex/Classes/jmb}

\bibliographystyle{agsm}
\renewcommand{\bibname}{References} % changes the header; default: Bibliography
\bibliography{7_references/references_myphd} % adjust this to fit your BibTex file
\end{small}

%\end{multicols}

% --------------------------------------------------------------
% Various bibliography styles exit. Replace above style as desired.

% in-text refs: (1) (1; 2)
% ref list: alphabetical; author(s) in small caps; initials last name; page(s)
%\bibliographystyle{Latex/Classes/PhDbiblio-case} % title forced lower case
%\bibliographystyle{Latex/Classes/PhDbiblio-bold} % title as in bibtex but bold
%\bibliographystyle{Latex/Classes/PhDbiblio-url} % bold + www link if provided

%\bibliographystyle{Latex/Classes/jmb} % calls style file jmb.bst
% in-text refs: author (year) without brackets
% ref list: alphabetical; author(s) in normal font; last name, initials; page(s)

%\bibliographystyle{plainnat} % calls style file plainnat.bst
% in-text refs: author (year) without brackets
% (this works with package natbib)


% --------------------------------------------------------------

% according to Dresden med fac summary has to be at the end
%
% Thesis Abstract -----------------------------------------------------


%\begin{abstractslong}    %uncommenting this line, gives a different abstract heading
\begin{abstracts}        %this creates the heading for the abstract page

Procedural content generation (PCG) – the process of generating data algorithmically – is a technique that has applications across a variety of domains. In the research outlined in this report, focus is directed to the use of PCG as a means to generate novel-like stories. The challenge is twofold: research into procedural generation methods, as well as the structure and language of stories, addressing questions such as: “What are the elements of a good story?”. Finally, methods to codify these elements must be investigated, in such a way that they can be used by a procedural generation technique, effectively combining the two disciplines.

\end{abstracts}
%\end{abstractlongs}


% ---------------------------------------------------------------------- 



\end{document}
