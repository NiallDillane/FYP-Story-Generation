% this file is called up by thesis.tex
% content in this file will be fed into the main document

\chapter{Implementation} % top level followed by section, subsection


% ----------------------- paths to graphics ------------------------

% change according to folder and file names
\ifpdf
    \graphicspath{{4_chapter4/figures/PNG/}{4_chapter4/figures/PDF/}{4_chapter4/figures/}}
\else
    \graphicspath{{4_chapter4/figures/EPS/}{4_chapter4/figures/}}
\fi


% ----------------------- contents from here ------------------------

\section{Web Scraping}

Gathered training data from reddit, few different approaches, different sized datasets produced.

\subsection{Reddit API}

Limited to 1000 posts

\subsection{PushShift API}

Third party data source of reddit posts, able to circumvent the limit. 

\subsection{Datasets}

Small for testing, large unwieldy, medium used. Found that perplexity score was worse with medium compared to small.


\section{Training the Model}

Worried about hardware, online resources to the rescue! Sample scripts for GPT-2 provided by huggingface

\subsection{HuggingFace}

Open source abstraction of GPT-2, convenient scripts that can be modified and used. Checkpoints stored at various points.

\subsubsection{Scripts}

script code and explanation

\subsection{Google Colab}

Online jupyter notebook environment, can use a hosted runtime to take advantage of GPU/TPU, free for 12 hours at a time. 

\subsubsection{Workflow}

Results stored in runtime and can be exported to Google Drive and downloaded (checkpoints useful here).


\section{Python Flask API}

API to encapsulate scripts, easily callable and customisable.


\section{React JS}

Web framework, modern JavaScript library for building UI.

\subsection{Hooks}

Functional approach, no classes, using state which is passed around. Challenging new way of thinking but extensible and clean.


%\lstset{language=Python}
%\lstset{basicstyle=\tiny, 
%breaklines=true, 
%rulesepcolor=\color{c++green}, 
%keywordstyle=\color{c++purple}\bfseries, 
%stringstyle=\color{c++red}, 
%commentstyle=\color{c++green},}  
%
%%Load your code file below
%\lstinputlisting[fontadjust=\true]{ahrs_triad_mote.c}


% ---------------------------------------------------------------------------
% ----------------------- end of thesis sub-document ------------------------
% ---------------------------------------------------------------------------