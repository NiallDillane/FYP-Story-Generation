% this file is called up by thesis.tex
% content in this file will be fed into the main document

%: ----------------------- name of chapter  -------------------------

\chapter{Related Research} % top level followed by section, subsection

%: ----------------------- paths to graphics ------------------------

% change according to folder and file names
\ifpdf
    \graphicspath{{2_realated_research/figures/PNG/}{2_realated_research/figures/PDF/}{2_realated_research/figures/}}
\else
    \graphicspath{{2_realated_research/figures/EPS/}{2_realated_research/figures/}}
\fi

%: ----------------------- contents from here ------------------------

Much research has been undertaken on this topic, with a variety of approaches.

\section{Good Stories}

First we must discuss the basic issue of what a good story even is, apart from the technology to produce one.

\subsection{Structure}

Structure was the first element to be researched, as the highest level of a story planning process. Popular methods include the Hero’s Journey (\Citealt{campbell2008hero}), which describes a cycle of sorts: a call to adventure, crossing from the known to the unknown, transformation etc. This is a more granular structure, most known for its application in Star Wars.

Another method is the three (or five) act structure (Trottier 1998), wherein the story is broken down into distinct sections, typically: setup, confrontation, resolution. This is a bit more versatile, which was something to consider for the later steps of “filling in the gaps” with PCG. There was a balance to be struck: providing some structure so that the generation has some level of cogency, but also allowing flexibility so that not all stories are the same.
Following this, a tree-based approach became appealing, at the higher levels resembling more of a novel plan with plot points, and is eventually fleshed out below into sentences and phrases. This is a familiar concept in Computer Science, and it seems like the best way to maintain a coherent story. Some optional user interaction would be ideal – e.g. “Tell me a story about a dog with superpowers” – a kind of “seed”. These could also be generated autonomously.


\subsection{Wiki page}

You can learn more about Latex by clicking \href{http://en.wikibooks.org/wiki/LaTeX}{here}.


\section{Another Section}

\subsection{Another Subsection}

\subsubsection{Another Subsubsection Title}




% ---------------------------------------------------------------------------
%: ----------------------- end of thesis sub-document ------------------------
% ---------------------------------------------------------------------------

