
% this file is called up by thesis.tex
% content in this file will be fed into the main document

%: ----------------------- introduction file header -----------------------
\chapter{Introduction}

% the code below specifies where the figures are stored
\ifpdf
    \graphicspath{{1_introduction/figures/PNG/}{1_introduction/figures/PDF/}{1_introduction/figures/}}
\else
    \graphicspath{{1_introduction/figures/EPS/}{1_introduction/figures/}}
\fi

% ----------------------------------------------------------------------
%: ----------------------- introduction content ----------------------- 
% ----------------------------------------------------------------------

In order to be able to automatically generate stories we first need to understand the structure of a story. Indeed, not only the structure that would make one grammatically correct, but also those devices, structures and styles that help make a good story. This is probably the largest challenge in the project, a formula which is still unclear after millennia of storytelling. 

Furthermore, there have been significant advancements over recent years in the Natural Language Processing (NLP) field, with new architectures and technologies appearing and outpacing each other. Choosing a direction which is sufficiently developed but still relatively state of the art will be difficult, especially since few of these have specifically tackled the topic of novel-like stories, instead focusing more on news article or script style content.

However, it is this rapid progress and uncertain nature that make the topic so fascinating. There has been great progress in images, video, even games, but text generation has lagged behind somewhat. This signals that language is more nuanced, difficult to replicate, and still in its infancy. 

\vspace{10mm}

% here we declare a new section

\section{Aims and Objectives} 

The main question that this dissertation addresses is .... 
In order to address this question, this work focuses on the following issues in the context of .... :

\begin{itemize}

\item Firstly, . . .

\item Secondly, . . . 

\end{itemize}



% here we declare a new section
\section{Methodology} 

In order to address this question the following approach was taken. 
 
 
% here we declare a new section
\section{Research Contribution}

The primary research contribution is: 





\section{Thesis Outline} 

The remaining chapters of this dissertation are as follows:


\emph{Chapter Two} is an introductory discussion to . . .

\emph{Chapter Three} is a historical review of  . . .

\emph{Chapter Four} presents the  . . . 

\emph{Chapter Five} describes the . . .

\emph{Chapter Six} draws conclusions and evaluates the . . .  Lastly, it suggests possible future works. 

\vspace{5 mm}

A series of documents have been included in the Appendix section of this dissertation. These are:

\begin{itemize}
\item \emph{Appendix A} outlines . . .

\item \emph{Appendix B} presents . . .

\item \emph{Appendix C} includes . . .
\end{itemize} 

\vspace{5 mm}

Attached to this dissertation is a CD containing the following items:

\begin{itemize}
\item \emph{folder 1}: . . .

\item \emph{folder 2}: . . .

\end{itemize}


% ----------------------------------------------------------------------



