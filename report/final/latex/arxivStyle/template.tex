\documentclass[twoside, 12pt]{article}


\usepackage{arxiv}

\usepackage[utf8]{inputenc} % allow utf-8 input
\usepackage[T1]{fontenc}    % use 8-bit T1 fonts
\usepackage{hyperref}       % hyperlinks
\usepackage{url}            % simple URL typesetting
\usepackage{booktabs}       % professional-quality tables
\usepackage{amsfonts}       % blackboard math symbols
\usepackage{nicefrac}       % compact symbols for 1/2, etc.
\usepackage{microtype}      % microtypography
\usepackage{lipsum}
\usepackage{apacite}
\usepackage{graphicx}
\graphicspath{ {./images/} }
\renewcommand{\baselinestretch}{1.5}

\title{Procedural Story Generation using Transformers}


\author{
  Niall Dillane \\
  Computer Science and Information Systems\\
  University of Limerick\\
  Limerick, Ireland \\
  \texttt{13132911@studentmail.ul.ie} \\
  %% examples of more authors
   \And
  James Patten \\
  Computer Science and Information Systems\\
  University of Limerick\\
  Limerick, Ireland \\
  \texttt{james.patten@ul.ie} \\
}

\begin{document}
\maketitle

\begin{abstract}
Procedural content generation (PCG) – the process of generating data algorithmically – is a technique that has applications across a variety of domains. In the research outlined in this report, focus is directed to the use of PCG as a means to generate novel-like stories. The challenge is twofold: research into procedural generation methods, as well as the structure and language of stories, addressing questions such as: “What are the elements of a good story?”. Finally, methods to codify these elements must be investigated, in such a way that they can be used by a procedural generation technique, effectively combining the two disciplines.
\end{abstract}


% keywords can be removed
\keywords{procedural generation \and algorithms \and AI \and language \and neural networks \and transformers}


\section{Introduction}
In order to be able to automatically generate stories we first need to understand the structure of a story. Indeed, not only the structure that would make one grammatically correct, but also those devices, structures and styles that help make a good story. This is probably the largest challenge in the project, a formula which is still unclear after millenia of storytelling. 

Furthermore, there have been significant advancements over recent years in the Natural Language Processing (NLP) field, with new architectures and technologies appearing and outpacing each other. Choosing a direction which is sufficiently developed but still relatively state of the art will be difficult, especially since few of these have specifically tackled the topic of novel-like stories, instead focusing more on news article or script style content.

However, it is this rapid progress and uncertain nature that make the topic so fascinating. There has been great progress in images, video, even games, but text generation has lagged behind somewhat. This signals that language is more nuanced, difficult to replicate, and still in its infancy. 



\section{Good Stories}
\label{sec:good_stories}

%\lipsum[4] See Section \ref{sec:headings}.

\subsection{Good Stories}
Structure was the first element to be researched, as the highest level of a story planning process. Popular methods include the Hero’s Journey \cite{campbell2008hero}, which describes a cycle of sorts: a call to adventure, crossing from the known to the unknown, transformation etc. This is a more granular structure, most known for its application in Star Wars. 

Another method is the three (or five) act structure \cite{trottier1998screenwriter}, wherein the story is broken down into distinct sections, typically: setup, confrontation, resolution. This is a bit more versatile, which was something to consider for the later steps of “filling in the gaps” with PCG. There was a balance to be struck: providing some structure so that the generation has some level of cogency, but also allowing flexibility so that not all stories are the same.

Following this, a tree-based approach became appealing, at the higher levels resembling more of a novel plan with plot points, and is eventually fleshed out below into sentences and phrases. This is a familiar concept in Computer Science, and it seems like the best way to maintain a coherent story. Some optional user interaction would be ideal – e.g. “Tell me a story about a dog with superpowers” – a kind of “seed”. These could also be generated autonomously.

\begin{figure}
	\centering
	\includegraphics{images/herosjourney.png}
	\caption{commons.wikimedia.org}
	\label{fig:herojourney}
\end{figure}


\subsubsection{Headings: third level}
\lipsum[6]

\paragraph{Paragraph}
The quick brown fox jumps over the lazy dog
\lipsum[7]

\section{Examples of citations, figures, tables, references}
\label{sec:others}
\lipsum[8] 

The documentation for \verb+natbib+ may be found at
\begin{center}
  \url{http://mirrors.ctan.org/macros/latex/contrib/natbib/natnotes.pdf}
\end{center}
Of note is the command \verb+\citet+, which produces citations
appropriate for use in inline text.  For example,
\begin{verbatim}
   \citet{hasselmo} investigated\dots
\end{verbatim}
produces
\begin{quote}
  Hasselmo, et al.\ (1995) investigated\dots
\end{quote}

\begin{center}
  \url{https://www.ctan.org/pkg/booktabs}
\end{center}


\subsection{Figures}
\lipsum[10] 
See Figure \ref{fig:fig1}. Here is how you add footnotes. \footnote{Sample of the first footnote.}
\lipsum[11] 

\begin{figure}
  \centering
  \fbox{\rule[-.5cm]{4cm}{4cm} \rule[-.5cm]{4cm}{0cm}}
  \caption{Sample figure caption.}
  \label{fig:fig1}
\end{figure}

\subsection{Tables}
\lipsum[12]
See awesome Table~\ref{tab:table}.

\begin{table}
 \caption{Sample table title}
  \centering
  \begin{tabular}{lll}
    \toprule
    \multicolumn{2}{c}{Part}                   \\
    \cmidrule(r){1-2}
    Name     & Description     & Size ($\mu$m) \\
    \midrule
    Dendrite & Input terminal  & $\sim$100     \\
    Axon     & Output terminal & $\sim$10      \\
    Soma     & Cell body       & up to $10^6$  \\
    \bottomrule
  \end{tabular}
  \label{tab:table}
\end{table}

\subsection{Lists}
\begin{itemize}
\item Lorem ipsum dolor sit amet
\item consectetur adipiscing elit. 
\item Aliquam dignissim blandit est, in dictum tortor gravida eget. In ac rutrum magna.
\end{itemize}


\bibliographystyle{apacite}  
\bibliography{references}  %%% Remove comment to use the external .bib file (using bibtex).
%%% and comment out the ``thebibliography'' section.

\end{document}
